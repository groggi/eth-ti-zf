\chapter{Endliche Automaten}
\begin{itemize}
  \item Endliche Automaten sind das einfachste Berechnungsmodell, welches in der Informatik betrachtet wird.
  \item Sie entsprechen speziellen Programmen, die Entscheidungsprobleme lösen.
  \item Endliche Automaten verwenden dabei keine Variabeln.
  \item Die Eingabe wird nur einmal von links nach rechts gelesen.
  \item Nach dem Lesen des letzten Buchstabens steht das Resultat sofort fest.
\end{itemize}

\section{Darstellung endlicher Automaten}
\begin{definition}
Ein (deterministischer) \textbf{endlicher Automat (EA)} ist ein Quintupel $M = (Q, \Sigma, \delta, q_0, F)$:
\begin{itemize}
  \item $Q$ ist eine endliche Menge von \textbf{Zuständen}
  \item $\Sigma$ ist ein Alphabet, welches als \textbf{Eingabealphabet} bezeichnet wird
  \item $q_0 \in Q$ ist der \textbf{Anfangszustand}
  \item $F \subseteq Q$ ist die \textbf{Menge der akzeptierenden Zustände}
  \item $\delta$ eine Funktion $\delta: Q \times \Sigma \to Q$, welche als \textbf{Übergangsfunktion} bezeichnet wird. Daher bedeutet $\delta(q_i, a) = p$, dass falls $M$ im Zustand $q_i \in Q$ den Buchstaben $a \in \Sigma$ liest, es in den Zustand $p \in Q$ übergeht.\\
\end{itemize}

\end{definition}

\begin{definition}
Eine \textbf{Konfiguration} von $M$ ist ein Element aus $Q \times \Sigma^*$. Falls $M$ in der Konfiguration $(q_i, w) \in Q \times \Sigma^*$ ist, so bedeutet es, dass $M$ aktuell im Zustand $q_i \in Q$ ist und noch den Suffix $w \in Sigma^*$ des Eingabeworts zu lesen hat.

Die Konfiguration $(q_0, x) \in \{q_0\} \times \Sigma^*$ nennt man eine \textbf{Startkonfiguration} von $M$ auf $x$. Die Berechnung des Worts $x$ beginnt somit an der Startkonfiguration $q_0$.\\
\end{definition}

\begin{definition}
Eine \textbf{Endkonfiguration} von $M$ hat die Form $(q_i, \lambda) \in Q \times \{ \lambda \}$\\
\end{definition}

\begin{definition}
Ein \textbf{Schritt} von $M$ ist eine Relation $\vdash_M \subseteq (Q \times \Sigma^*) \times (Q \times \Sigma^*)$ und ist definiert durch
\[
(q, w) \vdash_{M} (p, x) \Leftrightarrow w = ax,\ a \in \Sigma \land \delta(q, a) = p
\]

Es handelt sich somit um eine Relation auf Konfigurationen des EA $M$. Es beschreibt den Übergang von einer Konfiguration in die nächste, nachdem der nächste Buchstabe von $w$ (hier der Buchstabe $a$) gelesen wurde.\\
\end{definition}

\begin{definition}
Eine \textbf{Berechnung} $C$ von $M$ ist eine endliche Folge $C = C_0, C_1, C_2, C_3, \ldots, C_n$ von Konfigurationen ($C_i \in (Q \times \Sigma^*),\ i=0 \ldots n$), so dass $C_i \vdash_M C_{i+1}$ gilt für alle $0 \leq i \leq n-1$.

Falls $C_0 = (q_0, x)$ und $C_n \in Q \times \{ \lambda \}$, so ist $C$  die Berechnung von $M$ auf einer Eingabe $x \in \Sigma^*$.

Falls $C_n \in F \times \{ \lambda \}$, so sagen wir, dass $C$ eine \textbf{akzeptierende Berechnung} von $M$ auf $x$ ist, und dass $M$ das Wort $x$ akzeptiert.

Falls $C_n \in (Q - F) \times \{ \lambda \}$, so ist $C$ eine \textbf{verwerfende Berechnung} von $M$ auf $x$, und dass $M$ das Wort $x$ verwirft.\\
\end{definition}

\begin{definition}
Die von $M$ akzeptierte Sprache $L(M)$ ist definiert als
\[
L(M) := \{ w \in \Sigma^* | \text{ die Berechnung von $M$ auf $w$ endet in einer Endkonfiguration $(q, \lambda)$ mit $q \in F$} \}
\]
\end{definition}

\begin{definition}
$\mathcal{L}(EA) = \{L(M) | M \text{ ist ein EA} \}$ ist die Klasse der Sprachen, die von endlichen Automaten akzeptiert werden. $\mathcal{L}(EA)$ bezeichnet man auch als die \textbf{Klasse der regulären Sprachen}, und jede Sprache $L$ aus $\mathcal{L}(EA)$ als \textbf{regulär}.\\
\end{definition}

\begin{definition}
Sei $M$ ein EA. Wir definieren $\vdash_M^*$ als die reflexive und transitive Hülle der Schrittrelation $\vdash_M$ von $M$. Somit gilt
\[
(q, w) \vdash_M^* (p, u) \Leftrightarrow (q = p \land w = u) \lor \exists k \in \N - \{0\} \text{ so dass}
\]
\begin{itemize}
  \item $w = a_1 a_2 a_3 a_4 \ldots a_k u, \ a_i \in \Sigma \ i = 1, \ldots, k$ und
  \item $\exists r_1, r_2, \ldots, r_{k-1} \in Q$, so dass
  $(q, w) \vdash_M (r_1, a_2\ldots a_k u) \vdash_M (r_2, a_3 \ldots a_k u) \vdash_M \ldots \vdash_M (r_{k-1}, a_k u) \vdash_M (p, u)$
\end{itemize}

$(q, w) \vdash_M^* (p, u)$ sagt also aus, dass es eine Berechnung von $M$ gibt, die ausgehen von der Konfiguration $(q, w)$ zur Konfiguration $(p, u)$ führt.\\
\end{definition}

\begin{definition}
Sei $\hat{\delta}: Q \times \Sigma^* \to Q$ definiert durch:
\begin{itemize}
  \item $\hat\delta(q, \lambda) = q$ für alle $q \in Q$
  \item $\hat\delta(q, wa) = \delta(\hat\delta(q, w), a)$ für alle $a \in \Sigma,\ w \in \Sigma^*,\ q \in Q$
\end{itemize}

Wenn $M$ im Zustand $q$ ist und das Wort $w$ zu lesen beginnt, dann bedeutet $\hat\delta(q, w) = p$ dass $M$ im Zustand $p$ enden wird. Oder anders gesagt, es gilt: $(q, w) \vdash_M^* (p, \lambda)$.

Gekürzt können wir nun $L(M)$ wie folgt definieren:
\[
L(M) = \{ w \in \Sigma^* \ |\ (q_0, w) \vdash_M^* (p, \lambda),\ p \in F \} = \{ w \in \Sigma^* \ |\ \hat\delta(q_0, w) \in F \}
\]\\
\end{definition}
