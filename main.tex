\documentclass[a4paper,titlepage]{report}

%
% utf8 fix
%
\usepackage[utf8]{inputenc}

%
% amsthm
%
\usepackage{amsthm}

\theoremstyle{plain}
\newtheorem{lemma}{Lemma}[section]
\newtheorem{theorem}{Theorem}[section]
\newtheorem{corollary}{Korollar}[section]
\newtheorem{tipp}{Tipp}[section]

\theoremstyle{definition}
\newtheorem{definition}{Definition}[section]
\newtheorem{satz}{Satz}[section]
\newtheorem{hilfssatz}{Hilfssatz}[section]
\newtheorem{remark}{Bemerkung}[section]

%
% disable paragraph indentation
%
\usepackage{parskip}

%
% PGF/TikZ for state machines
%
\usepackage{
  pgf,
  tikz
}
\usetikzlibrary{
  arrows,
  automata
}


%
% other packages
%
\usepackage{
  amssymb,
  amsfonts,
  amsmath,
  enumitem,
  hyperref,
  todonotes
}

%
% listings
%
\usepackage{listings}

\lstset{
  language=Pascal,
  escapeinside={||}
}

%
% number sets
%
\newcommand{\R}{\mathbb{R}}
\newcommand{\Z}{\mathbb{Z}}
\newcommand{\N}{\mathbb{N}}
\newcommand{\Q}{\mathbb{Q}}
\newcommand{\C}{\mathbb{C}}
\newcommand{\F}{\mathbb{F}}
\newcommand{\E}{\mathbb{E}}
\newcommand{\LL}{\mathcal{L}}
\newcommand{\powerset}{\mathcal P}

%
% section settings
%
\setcounter{secnumdepth}{3}

%
% document
%

\begin{document}

% title
\title{Theoretische Informatik Zusammenfassung HS13}
\author{Gregor Wegberg}
\date{\today}
\maketitle

% index
\tableofcontents
\clearpage

% todo list
\listoftodos
\clearpage

% chapters
\chapter{Alphabete, Wörter, Sprachen}
Zur Darstellung von Daten werden Symbole verwendet. Diese wiederum werden zur Bildung von Wörtern genutzt. Und aus Wörtern bildet sich eine Sprache.

\section{Alphabet}
\begin{definition}
Eine endliche nichtleere Menge $\Sigma$ heisst \textbf{Alphabet}. Die Elemente von $\Sigma$ werden als \textbf{Buchstaben, Zeichen oder Symbole} bezeichnet.
\end{definition}

Häufig benötigte Alphabete:
\begin{itemize}
    \item $\Sigma_\text{bool} = \{0,1\}$
    \item $\Sigma_\text{lat} = \{a, b, c, \ldots, z\}$
    \item $\Sigma_\text{Tastatur} = \Sigma_\text{lat} \cup \{A, B, C, \ldots, \textvisiblespace, <, >, \ldots\}$ (Alphabet aller Zeichen auf einer Tastatur)
    \item $\Sigma_m = \{0, 1, 2, \ldots, m - 1\}$ (Alphabet für die $m$-adische Darstellung von Zahlen, $m \geq 1$)
    \item $\Sigma_\text{logic} = \{0, 1, x, (, ), \land, \lor, \lnot\}$ (Alphabet um Boole'sche Formeln darzustellen)
\end{itemize}

\section{Wort}
\subsection{Grundlagen}
Ein \textbf{Wort} wird aus Elementen eines zugrundeliegenden Alphabets gebildet.\\

\begin{definition}
Sei $\Sigma$ ein Alphabet. Ein \textbf{Wort} über $\Sigma$ ist eine endliche Folge von Buchstaben aus $\Sigma$.\\
\end{definition}

\begin{remark}
$\Sigma^*$ ist die Menge aller Wörter über dem Alphabet $\Sigma$. $\Sigma^+ = \Sigma^* - \{\lambda\}$, also die Menge aller Wörter ohne das leere Wort.\\
\end{remark}


\begin{definition}
Das \textbf{leere Wort} wird durch $\lambda$ (oft auch $\epsilon$) dargestellt und entspricht der leeren Folge.\\
\end{definition}

\begin{definition}
Die \textbf{Länge} eines Wortes $w$, bezeichnet durch $|w|$, ist die Länge der Folge, d.h. die Anzahl vorkommender Buchstaben in $w$.\\
\end{definition}

\begin{remark}
Das leere Wort $\lambda$ ist ein Wort über jedem Alphabet.\\
\end{remark}

\begin{definition}
Sei $w \in \Sigma^*$ und $a \in \Sigma$. Dann ist $|w|_a$ definiert als die Anzahl der Vorkommen von $a$ in $w$.
\end{definition}


\subsection{Beispiele für Kodierungen}
\subsubsection{Natürliche Zahlen}
Sei $m$ eine natürliche Zahl. Dann erzeugt die Funktion $\operatorname{Bin}(m) \in \Sigma_\text{bool}^*$ die binäre Darstellung der natürlichen Zahl $m$. Damit $\operatorname{Bin}(m)$ eindeutig ist, soll die Funktion die kürzeste binäre Darstellung liefern (das erste Zeichen ist eine 1).

Die Umkehrfunktion ist $\text{Nummer}(x) = \sum_{i = 1}^n x_i \cdot 2^{n-i}$ und erzeugt für eine binäre Darstellung $x$ die natürliche Zahl.

\ subsubsection{Graphen}
Gerichtete Graphen $G = (V, E)$ ($V$ ist die Knotenmenge, $E$ die Kantenmenge) können durch eine Adjazenzmatrix $M_G$ beschrieben werden: $M_G = [a_{ij}]$. Falls der Knoten $v_i \in V$ mit dem Knoten $v_j \in V$ verbunden ist, so gilt für $M_G$: $a_{ij} = 1$, sonst $a_{ij} = 0$. Die Adjazenzmatrix kann nun durch ein Wort über dem Alphabet $\Sigma = \{0, 1, \#\}$ beschrieben werden. Dazu schreibt man den Inhalt jeder Zeile nacheinander und trennt die Zeilen im entstehenden Wort durch $\#$.

Möchte man einen gewichteten Graphen $G = (V, E, h)$ mit einer Funktion $h(e) \in \N - \{0\}$ für eine Kante $e \in E$, so kann dies ebenfalls über dem Alphabet $\Sigma = \{0, 1, \#\}$ gemacht werden. Wieder schreibt man den Inhalt jeder Zeile der Adjazenzmatrix nacheinander. Dabei kodiert man die Gewichtung mittels $\operatorname{Bin}(h(e))$, trennt die einzelnen Matrixeinträge durch $\#$ ab und Zeilen durch $\#\#$.

\subsubsection{Bool'sche Formeln}
Für Bool'sche Formeln verwenden wir das Alphabet $\Sigma_\text{logic} = \{0, 1, x, (, ), \land, \lor, \lnot\}$. In Bool'schen Formeln kommen, im Gegensatz zu unserem Alphabet, beliebig viele Variabeln $x_i$ vor. Unser Alphabet muss aber gleichzeitig endlich sein. Deshalb werden die Variabeln $x_i$ durch das Wort $x\operatorname{Bin}(i)$ kodiert. Die restlichen Symbole können direkt übernommen werden.

\subsection{Konkatenation}
\begin{definition}
Die \textbf{Verkettung (Konkatenation)} für ein Alphabet $\Sigma$ ist eine Abbildung $K: \Sigma^* \times \Sigma^* \to \Sigma^*$, so dass für alle $x, y \in \Sigma^*$
\[
K(x, z) = x \cdot y = xy
\]
gilt.\\
\end{definition}

\begin{remark}
Die Verkettung $K$ über $\Sigma$ ist eine assoziative Operation:
\[
K(u, K(v, w)) = u \cdot (v \cdot w) = u \cdot (vw) = uvw = (u \cdot v) \cdot w = K(K(u, v), w)
\]\\
\end{remark}

\begin{remark}
Für jedes $w \in \Sigma^*$ gilt
\[
w \cdot \lambda = \lambda \cdot w = x
\]\\
\end{remark}

\begin{remark}
Die Konkatenation ist nur für einelementige Alphabete kommutativ.\\
\end{remark}

\begin{remark}
Für alle $x, y \in \Sigma^*$ gilt:
\[
|xy| = |x \cdot y| = |x| + |y|
\]
\end{remark}

\begin{definition}
Sei $\Sigma$ ein Alphabet. Für alle $x \in \Sigma^*$ und alle $i \in \N$ wird die $i$-te Iteration $x^i$ von $x$ definiert als:
\[
x^0 = \lambda,\quad x^1 = x,\quad x^i = x \cdot x^{i-1}
\]
\end{definition}

\subsection{Teilworte}
\begin{definition}
Seien $u, w \in \Sigma^*$ für ein Alphabet $\Sigma$.
\begin{itemize}
  \item $v$ ist ein \textbf{Teilwort} von $w$ $\Leftrightarrow \exists x, y \in \Sigma^*: w = xvy$
  \item $v$ ist ein \textbf{Suffix} von $w$ $\Leftrightarrow \exists x \in \Sigma^*: w = xv$
  \item $v$ ist ein \textbf{Präfix} von $w$ $\Leftrightarrow \exists x \in \Sigma^*: w = vx$
  \item $v$ ist ein \textbf{echtes} Teilwort/Suffix/Präfix von $w$ genau dann, wenn $v \not= w$, $v \not= \lambda$ und $v$ ist ein Teilwort/Suffix/Präfix von $w$
\end{itemize}

\end{definition}

\subsection{Ordnung}
\begin{definition}
Sei $\Sigma = \{s_1, s_2, \ldots, s_m\}$ ein Alphabet für ein beliebiges $m \geq 1$. Weiter sei $s_1 < s_2 < s_3 < \ldots < s_m$ eine Ordnung auf $\Sigma$. Darauf basierend wir die \textbf{kanonische Ordnung} auf $\Sigma^*$ für $u, v \in \Sigma^*$ wie folgt definiert:
\begin{align*}
u < v \Leftrightarrow \quad & |u| < |v|\\
&\lor (|u| = |v| \land u = x \cdot s_i \cdot u' \land v = x \cdot s_j \cdot v'\\
&\text{für beliebige } x, u', v' \in \Sigma^* \text{und } i < j)
\end{align*}

\end{definition}

\section{Sprache}
Eine Sprache wird durch eine beliebige Menge von Wörtern über einem festen Alphabet gebildet.

\begin{definition}
Eine \textbf{Sprache} $L$ über einem Alphabet $\Sigma$ ist eine Teilmenge von $\Sigma^*$.\\
\end{definition}

\begin{itemize}
  \item $L_\emptyset = \emptyset$ ist die leere Sprache (hat keine Elemente)
  \item $L_\lambda = \{\lambda\}$ ist die einelementige Sprache, die nur das leere Wort enthält\\
\end{itemize}

\begin{definition}
Sind $L_1$ und $L_2$ zwei Sprachen über demselben Alphabet $\Sigma$, so ist
\[
L_1 \cdot L_2 = L_1 L_2 = \{vw | v \in L_1, w \in L_2\}
\]
die \textbf{Konkatenation} von $L_1$ und $L_2$.\\
\end{definition}

\begin{definition}
Ist $L$ eine Sprache über $\Sigma$, so wird definiert:
\begin{itemize}
  \item $L^0 = L_\lambda$
  \item $L^{i+1} = L^i \cdot L, \quad \forall i \in \N$
  \item $L^* = \bigcup_{i \in \N} L^i$ (ist der \textbf{Kleene'sche Stern})
  \item $L^+ = \bigcup_{i \in \N - \{0\}} L^i = L \cdot L^*$\\
\end{itemize}
\end{definition}

\begin{remark}
\
\begin{itemize}
  \item $\Sigma^i = \{w | w \in \Sigma^* \land |w| = i\}$
  \item $L_\emptyset L = L_\emptyset = \emptyset$
  \item $L_\lambda L = L$\\
\end{itemize}

\end{remark}

\begin{lemma}
Seien $L_1, L_2, L_3$ Sprachen über dem Alphabet $\Sigma$. Dann gilt $L_1 L_2 \cup L_1 L_3 = L_1 (L_2 \cup L_3)$.\\
\end{lemma}

\begin{lemma}
Seien $L_1, L_2, L_3$ Sprachen über dem Alphabet $\Sigma$. Dann gilt $L_1 (L_2 \cap L_3) \subseteq L_1 L_2 \cap L_1 L_3$\\
\end{lemma}

\begin{definition}
Seien $\Sigma_1, \Sigma_2$ zwei Alphabete. Ein \textbf{Homomorphismus} von $\Sigma_1^*$ nach $\Sigma_2^*$ ist jede Funktion $h: \Sigma_1^* \to \Sigma_2^*$ mit folgenden Eigenschaften:
\begin{itemize}
  \item $h(\lambda) = \lambda$
  \item $h(uv) = h(u) \cdot h(v) \quad \forall u,v \in \Sigma_1^*$\\
\end{itemize}
\end{definition}

\begin{remark}
Um einen Homomorphismus zu spezifizieren reicht es aus für alle Zeichen $a \in \Sigma_1$ $h(a)$ zu definieren.
\end{remark}

\section{Algorithmische Probleme}
Ein Programm ist im Grunde eine Abbildung $A$, welche ein Wort über $\Sigma_1$ in ein Wort über $\Sigma_2$ abbildet: $A: \Sigma_1^* \to \Sigma_2^*$. Somit ist sowohl die Eingabe, wie auch die Ausgabe des Programms als Wort kodiert und $A$ bestimmt für jedes Eingabewort ein bestimmtes Ausgabewort.


Zwei Programme $A$ und $B$ sind \textbf{äquivalent}, wenn für alle $x \in \Sigma_1^*$ $A(x) = B(x)$ gilt.

\subsection{Entscheidungsproblem}
\begin{definition}
Das \textbf{Entscheidungsproblem $(\Sigma, L)$} für ein gegebenes Alphabet $\Sigma$ und eine gegebene Sprache $L \subseteq \Sigma^*$ ist, für jedes $x \in \Sigma^*$ zu entscheiden ob
\[
x \in L \text{ oder } x \not\in L.
\]\\
\end{definition}

\begin{definition}
Ein Algorithmus $A$ \textbf{löst} das Entscheidungsproblem $(\Sigma, L)$, falls für alle $x \in \Sigma^*$ gilt:
\[
A(x) =
\begin{cases}
1, &\text{falls } x \in L \\
0, &\text{falls } x \not \in L
\end{cases}
\]
In diesem Fall sagt man, dass $A$ die Sprache $L$ \textbf{erkennt}.\\
\end{definition}

\begin{definition}
Seien $\Sigma, \Gamma$ zwei Alphabete. Wir sagen, dass ein Algorithmus $A$ eine \textbf{Funktion $f: \Sigma^* \to \Gamma^*$} berechnet, falls
\[
\forall x \in \Sigma^*: A(x) = f(x)
\]\\
\end{definition}

Das Entscheidungsproblem ist ein Spezialfall einer Funktionsberechnung.

\subsection{Relationsproblem}
\begin{definition}
Seien $\Sigma, \Gamma$ zwei Alphabete, und sei $R \subseteq \Sigma^* \times \Gamma^*$ eine Relation. Ein Algorithmus $A$ löst das \textbf{Relationsproblem} $R$, falls für jedes $x \in \Sigma^*$ gilt:
\[
(x, A(x)) \in R
\]
\end{definition}

\subsection{Optimierungsproblem}
\begin{definition}
Ein \textbf{Optimierungsproblem} ist ein 6-Tupel $\mathcal{U} = (\Sigma_I, \Sigma_O, L, \mathcal{M}, \operatorname{cost}, \operatorname{goal})$:
\begin{itemize}
  \item $\Sigma_I$ ist das Eingabealphabet
  \item $\Sigma_O$ ist das Ausgabealphabet
  \item $L \subseteq \Sigma_I^*$ ist die Sprache der zulässigen Eingaben
  \item $\mathcal{M}$ ist eine Funktion $\mathcal{M}: L \to \mathcal{P}(\Sigma_O^*)$. Für jedes $x \in L$ ist $\mathcal{M}(x)$ die Menge der zulässigen Lösungen für $x$
  \item $\operatorname{cost}$ ist eine Funktion $\operatorname{cost}: \bigcup_{x \in L}(\mathcal{M} \times \{x\}) \to \R^+$ und ist die Preisfunktion
  \item $\operatorname{goal} \in \{\text{Minimum}, \text{Maximum}\}$ ist das Optimierungsziel
\end{itemize}

Eine zulässige Lösung $\alpha \in \mathcal{M}(x)$ heisst \textbf{optimal} für den Problemfall $x$ des Optimierungsproblems $U$, falls
\[
\operatorname{cost}(\alpha, x) = \operatorname{Opt}_\mathcal{U} = \operatorname{goal}\{\operatorname{cost}(\beta, x) | \beta \in \mathcal{M}(x)\}
\]\\
\end{definition}

\begin{definition}
Ein Algorithmus $A$ \textbf{löst} $\mathcal{U}$, falls für jedes $x \in L$:
\begin{enumerate}
  \item $A(x) \in \mathcal{M}(x)$ ($A(x)$ ist eine zulässige Lösung des Problemfalls $x$ von $\mathcal{U}$)
  \item $\operatorname{cost}(A(x), x) = \operatorname{goal}\{\operatorname{cost}(\beta, x) | \beta \in \mathcal{M}(x)\}$
\end{enumerate}

\end{definition}

\begin{remark}
Oft wird die Spezifikation von $\Sigma_I, \Sigma_O$ bei Optimierungsproblemen weggelassen. Man geht davon aus, dass die verwendeten Daten kodiert werden können für ein $\Sigma_I, \Sigma_O$. So bleiben noch vier Dinge übrig, die spezifiziert werden müssen:
\begin{enumerate}
  \item die Menge der Problemfälle $L$, also die zulässigen Eingaben
  \item die Menge der Einschränkungen, gegeben durch jeden Problemfall $x \in L$, und damit $\mathcal{M}(x)$ für jedes $x \in L$. $\mathcal{M}(x)$ gibt uns Lösungen, die den Einschränkungen genügen für ein gegebenes $x \in L$
  \item die Kostenfunktion
  \item das Optimierungsziel
\end{enumerate}

\end{remark}

\subsubsection{Beispiel: Traveling Salesman Problem (TSP)}
\begin{description}
  \item[Eingabe:] Ein gewichteter Graph $(G, c)$, wobei $G = (V, E)$ ein Graph ist und $c: E \to \N - \{0\}$ die Kostenfunktion. Strikt formal müsste man das Eingabealphabet eingeben und mit diesem den Graphen kodieren.
  \item[Einschränkungen:] Für jeden Problemfall $(G, c)$ ist $\mathcal{M}(G, c)$ die Menge aller Hamiltonscher Kreise von $G$ mit der Kostenfunktion $e$.
  \item[Kosten:] Für jeden Hamiltonschen Kreis $H = v_{i_1}, v_{i_2}, \ldots, v_{i_n}, v_{i_1} \in \mathcal{M}(G, c)$:
    \[
    \operatorname{cost}((v_{i_1}, v_{i_2}, \ldots, v_{i_n}, v_{i_1}), (G, c)) = \sum_{j = 1}^n c \left (\{v_{i_j}, v_{i_{(j \mod n) + 1}}\} \right )
    \]
    Die Kosten jedes Hamiltonschen Kreises ist somit die Summe der Gewichte der besuchten Kanten.
  \item[Ziel:] Minimum\\
\end{description}

\begin{definition}
Ein Optimierungsproblem $\mathcal{U}_1 = (\Sigma_I, \Sigma_O, L', \mathcal{M}, \operatorname{cost}, \operatorname{goal})$ ist ein \textbf{Teilproblem} vom Optimierungsproblem $\mathcal{U}_2 = (\Sigma_I, \Sigma_O, L, \mathcal{M}, \operatorname{cost}, \operatorname{goal})$, falls $L' \subseteq L$.
\end{definition}

\subsubsection{Knotenüberdeckung}
\label{sec:knotenueberdeckung}
\begin{definition}
Eine \textbf{Knotenüberdeckung} eines Graphen $G = (V, E)$ ist jede Knotenmenge $U \subseteq V$, so dass jede Kante aus $E$ mit mindestens einem Knoten aus $U$ inzident ist. Eine Kante $\{u, v\} \in E$ ist inzident zu $u$ und $v$.

Anders gesagt: Jede Kante muss an mindestens einem Ende in einem Knoten enden, der in der Knotenmenge der Knotenüberdeckung ist.
\end{definition}

\subsubsection{Beispiel: Maximale Clique Problem (MAX-CL)}
\begin{definition}
Eine Clique eines Graphen $G = (V, E)$ ist jede Teilmenge $U \subseteq V$, so dass $\{\{u, v\} | u, v \in V, u \not= v\} \subseteq E$ (die Knoten von $U$ bilden einen vollständigen Teilgraphen von $G$).
\end{definition}

Das Maximale Clique Problem besteht nun darin eine Clique mit maximaler Kardinalität zu finden. Wir suchen also einen Teilgraphen mit maximaler Anzahl von Knoten.

\begin{description}
  \item[Eingabe:] Ein (ungerichteter) Graph $G = (V, E)$
  \item[Einschränkung:] $\mathcal{M}(G) = \{S \subseteq V | \{\{u, v\} | u, v \in S, u \not= v\} \subseteq E\}$
  \item[Kosten:] Für jedes $S \in \mathcal{M}(G)$ ist $\operatorname{cost}(S, G) = |S|$
  \item[Ziel:] Maximum
\end{description}


\chapter{Endliche Automaten}
\begin{itemize}
  \item Endliche Automaten sind das einfachste Berechnungsmodell, welches in der Informatik betrachtet wird.
  \item Sie entsprechen speziellen Programmen, die Entscheidungsprobleme lösen.
  \item Endliche Automaten verwenden dabei keine Variabeln.
  \item Die Eingabe wird nur einmal von links nach rechts gelesen.
  \item Nach dem Lesen des letzten Buchstabens steht das Resultat sofort fest.
\end{itemize}

\section{Darstellung endlicher Automaten}
\begin{definition}
Ein (deterministischer) \textbf{endlicher Automat (EA)} ist ein Quintupel $M = (Q, \Sigma, \delta, q_0, F)$:
\begin{itemize}
  \item $Q$ ist eine endliche Menge von \textbf{Zuständen}
  \item $\Sigma$ ist ein Alphabet, welches als \textbf{Eingabealphabet} bezeichnet wird
  \item $q_0 \in Q$ ist der \textbf{Anfangszustand}
  \item $F \subseteq Q$ ist die \textbf{Menge der akzeptierenden Zustände}
  \item $\delta$ eine Funktion $\delta: Q \times \Sigma \to Q$, welche als \textbf{Übergangsfunktion} bezeichnet wird. Daher bedeutet $\delta(q_i, a) = p$, dass falls $M$ im Zustand $q_i \in Q$ den Buchstaben $a \in \Sigma$ liest, es in den Zustand $p \in Q$ übergeht.\\
\end{itemize}

\end{definition}

\begin{definition}
Eine \textbf{Konfiguration} von $M$ ist ein Element aus $Q \times \Sigma^*$. Falls $M$ in der Konfiguration $(q_i, w) \in Q \times \Sigma^*$ ist, so bedeutet es, dass $M$ aktuell im Zustand $q_i \in Q$ ist und noch den Suffix $w \in Sigma^*$ des Eingabeworts zu lesen hat.

Die Konfiguration $(q_0, x) \in \{q_0\} \times \Sigma^*$ nennt man eine \textbf{Startkonfiguration} von $M$ auf $x$. Die Berechnung des Worts $x$ beginnt somit an der Startkonfiguration $q_0$.\\
\end{definition}

\begin{definition}
Eine \textbf{Endkonfiguration} von $M$ hat die Form $(q_i, \lambda) \in Q \times \{ \lambda \}$\\
\end{definition}

\begin{definition}
Ein \textbf{Schritt} von $M$ ist eine Relation $\vdash_M \subseteq (Q \times \Sigma^*) \times (Q \times \Sigma^*)$ und ist definiert durch
\[
(q, w) \vdash_{M} (p, x) \Leftrightarrow w = ax,\ a \in \Sigma \land \delta(q, a) = p
\]

Es handelt sich somit um eine Relation auf Konfigurationen des EA $M$. Es beschreibt den Übergang von einer Konfiguration in die nächste, nachdem der nächste Buchstabe von $w$ (hier der Buchstabe $a$) gelesen wurde.\\
\end{definition}

\begin{definition}
Eine \textbf{Berechnung} $C$ von $M$ ist eine endliche Folge $C = C_0, C_1, C_2, C_3, \ldots, C_n$ von Konfigurationen ($C_i \in (Q \times \Sigma^*),\ i=0 \ldots n$), so dass $C_i \vdash_M C_{i+1}$ gilt für alle $0 \leq i \leq n-1$.

Falls $C_0 = (q_0, x)$ und $C_n \in Q \times \{ \lambda \}$, so ist $C$  die Berechnung von $M$ auf einer Eingabe $x \in \Sigma^*$.

Falls $C_n \in F \times \{ \lambda \}$, so sagen wir, dass $C$ eine \textbf{akzeptierende Berechnung} von $M$ auf $x$ ist, und dass $M$ das Wort $x$ akzeptiert.

Falls $C_n \in (Q - F) \times \{ \lambda \}$, so ist $C$ eine \textbf{verwerfende Berechnung} von $M$ auf $x$, und dass $M$ das Wort $x$ verwirft.\\
\end{definition}

\begin{definition}
Die von $M$ akzeptierte Sprache $L(M)$ ist definiert als
\[
L(M) := \{ w \in \Sigma^* | \text{ die Berechnung von $M$ auf $w$ endet in einer Endkonfiguration $(q, \lambda)$ mit $q \in F$} \}
\]
\end{definition}

\begin{definition}
$\mathcal{L}(EA) = \{L(M) | M \text{ ist ein EA} \}$ ist die Klasse der Sprachen, die von endlichen Automaten akzeptiert werden. $\mathcal{L}(EA)$ bezeichnet man auch als die \textbf{Klasse der regulären Sprachen}, und jede Sprache $L$ aus $\mathcal{L}(EA)$ als \textbf{regulär}.\\
\end{definition}

\begin{definition}
Sei $M$ ein EA. Wir definieren $\vdash_M^*$ als die reflexive und transitive Hülle der Schrittrelation $\vdash_M$ von $M$. Somit gilt
\[
(q, w) \vdash_M^* (p, u) \Leftrightarrow (q = p \land w = u) \lor \exists k \in \N - \{0\} \text{ so dass}
\]
\begin{itemize}
  \item $w = a_1 a_2 a_3 a_4 \ldots a_k u, \ a_i \in \Sigma \ i = 1, \ldots, k$ und
  \item $\exists r_1, r_2, \ldots, r_{k-1} \in Q$, so dass
  $(q, w) \vdash_M (r_1, a_2\ldots a_k u) \vdash_M (r_2, a_3 \ldots a_k u) \vdash_M \ldots \vdash_M (r_{k-1}, a_k u) \vdash_M (p, u)$
\end{itemize}

$(q, w) \vdash_M^* (p, u)$ sagt also aus, dass es eine Berechnung von $M$ gibt, die ausgehen von der Konfiguration $(q, w)$ zur Konfiguration $(p, u)$ führt.\\
\end{definition}

\begin{definition}
Sei $\hat{\delta}: Q \times \Sigma^* \to Q$ definiert durch:
\begin{itemize}
  \item $\hat\delta(q, \lambda) = q$ für alle $q \in Q$
  \item $\hat\delta(q, wa) = \delta(\hat\delta(q, w), a)$ für alle $a \in \Sigma,\ w \in \Sigma^*,\ q \in Q$
\end{itemize}

Wenn $M$ im Zustand $q$ ist und das Wort $w$ zu lesen beginnt, dann bedeutet $\hat\delta(q, w) = p$ dass $M$ im Zustand $p$ enden wird. Oder anders gesagt, es gilt: $(q, w) \vdash_M^* (p, \lambda)$.

Gekürzt können wir nun $L(M)$ wie folgt definieren:
\[
L(M) = \{ w \in \Sigma^* \ |\ (q_0, w) \vdash_M^* (p, \lambda),\ p \in F \} = \{ w \in \Sigma^* \ |\ \hat\delta(q_0, w) \in F \}
\]\\
\end{definition}

\chapter{Berechenbarkeit}
\section{Die Methode der Diagonalisierung}
\begin{definition}
Seien \(A, B\) zwei Mengen.
\begin{itemize}
  \item \(|A| \leq |B|\), falls eine injektive Funktion \(f: A \to B\) existiert.
  \item \(|A| = |B|\), falls eine eine Bijektion \(f: A \to B\) bzw. \(f': B \to A\) existiert. Dies ist gleichbedeutend mit der Anforderung, dass \(|A| \leq |B| \land |B| \leq |A|\) ist.
  \item \(|A| < |B|\), falls \(|A| \leq |B|\) und es existiert keine injektive Abbildung \(f: B \to A\).\\
\end{itemize}
\end{definition}

\begin{definition}
Eine Menge \(A\) ist \textbf{abzählbar}, falls A endlich ist oder \(|A| = |\N|\).

Anders definiert: \(A\) ist abzählbar \(\Leftrightarrow\) es existiert eine injektive Funktion \(f: A \to \N\).
\end{definition}

Intuitive bedeutet die Abzählbarkeit, dass die Elemente der abzählbaren Menge nummeriert werden können. Sie führt also eine lineare Ordnung ein.

\begin{lemma}
Sei \(\Sigma\) ein beliebiges Alphabet. Dann ist \(\Sigma^*\) abzählbar.\\
\end{lemma}

\begin{satz}
Die Menge der Turingmaschinenkodierungen \(\operatorname{KodTM}\) ist abzählbar.\\
\end{satz}

\begin{lemma}
\( |\Q^+| = |\N| \). Somit ist die Menge der postiven rationalen Zahlen abzählbar. \\
\end{lemma}

\begin{lemma}
\( (\N - \{0\}) \times (\N - \{0\}) \) ist abzählbar.\\
\end{lemma}

\begin{satz}
\( [0, 1] \subseteq \R \) ist nicht abzählbar.\\
\end{satz}

\begin{satz}
\( \mathcal{P}(\Sigma_\text{bool}^*) \) ist nicht abzählbar.\\
\end{satz}

\begin{corollary}
\( |\operatorname{KodTM}| < |\mathcal{P}(\Sigma_\text{bool}^*)| \) und somit existieren unendlich viele nicht rekursiv abzählbare Sprachen über \( \Sigma_\text{bool} \).
\end{corollary}

\subsection{Diagonalsprache}
Sei \( w_i \) das \(i\)-te Wort in kanonischer Ordnung über \( \Sigma_\text{bool} \). Sei \(M_i\) die \(i\)-te Touringmaschine. Daraus definieren wir die Matrix \( A = [d_{ij}]_{i,j = 1, \ldots, \infty} \), wobei \( d_{ij} = 1 \Leftrightarrow M_i \text{ akzeptiert } w_j \).

Somit bestimmt die \(i\)-te Zeile der Matrix \(A\) die Sprache \( L(M_i) = \{w_j |\ d_{ij} = 1 \text{ für alle } j \in \N - \{0\}\} \).

Nun lässt sich durch die Idee der Diagonalisierung eine Sprache erzeugen, die keiner der Sprachen \(L(M_i)\) entspricht.
\begin{align*}
L_\text{diag} &= \{ w \in \Sigma_\text{bool}^* | w = w_i \text{ für ein } i \in \N - \{0\} \land M_i \text{ akzeptiert } w_i \text{ nicht} \}\\
&= \{ w \in \Sigma_\text{bool}^* | w = w_i \text{ für ein } i \in \N - \{0\} \land d_{ii} = 0 \}
\end{align*}

\begin{satz}
\( L_\text{diag} \notin \mathcal{L}_\text{RE} \)
\end{satz}

\section{Die Methode der Reduktion}
\begin{definition}
Seien \(L_1 \subseteq \Sigma_1^*, L_2 \subseteq \Sigma_2^*\) zwei Sprachen. \textbf{\( L_1 \) ist auf \( L_2 \) rekursiv reduzierbar}, \( L_1 \leq_R L_2 \), falls \[ L_2 \in \mathcal{L}_R \Rightarrow L_1 \in \mathcal{L}_R \]
\end{definition}

Die Intuition hinter der Bezeichnung \( L_1 \leq_R L_2 \) ist, dass \( L_2 \) bezüglich der algorithmischen Lösbarkeit mindestens so schwer wie \( L_1 \) ist.

Wäre also \( L_2 \) algorithmisch lösbar, sprich es gibt eine TM \(M\) mit \(L(M) = L_2\), dann wäre auch \(L_1\) durch eine TM \(M'\) lösbar. Die TM \(M'\) könnte mit Hilfe von \(M\) konstruiert werden.

\subsection{EE-Reduktion}
Das Ziel ist es nun für zwei Sprachen zu zeigen, dass \(L_1 \leq_R L_2\) gilt. Dazu sucht man eine TM \(M\), die für jede Eingabe \(x\) und das Entscheidungsproblem \((\Sigma_1, L_1)\) eine Ausgabe \(y\) generiert. Diese Ausgabe wiederum ist die Eingabe für das Entscheidungsproblem \((\Sigma_2, L_2)\).  Dieses Umschreiben soll dabei den Effekt haben, dass die Lösung des Entscheidungsproblems \((\Sigma_2, L_2)\) auf \(y\) gerade der Lösung des Entscheidungsproblems \((\Sigma_1, L_1)\) auf \(x\) entspricht.\\

\begin{definition}
Seien \(L_1 \subseteq \Sigma_1^*, L_2 \subseteq \Sigma_2^*\) zwei Sprachen. \(L_1\) ist auf \(L_2\) \textbf{EE-reduzierbar}, \(L_1 \leq_{EE} L_2\), wenn eine TM \(M\) existiert, die eine Abbildung \(f_M: \Sigma_1^* \to \Sigma_2^*\) für alle \(x \in \Sigma_1^*\) berechnet mit der Eigenschaft, dass \[x \in L_1 \Leftrightarrow f_M(x) \in L_2.\] In einem solchen Fall spricht man davon, dass die TM \(M\) die Sprache \(L_1\) auf die Sprache \(L_2\) reduziert.\\
\end{definition}

\begin{lemma}
Seien \( L_1 \subseteq \Sigma_1^*, L_2 \subseteq \Sigma_2^* \) zwei Sprachen. Falls \( L_1 \leq_{EE} L_2 \), dann auch \( L_1 \leq_R L_2 \).\\
\end{lemma}

\begin{lemma}
Sei \( \Sigma \) ein Alphabet. Für jede Sprache \( L \subseteq \Sigma^* \) gilt: \[ L \leq_R L^C \land L^C \leq_R L \]\\
\end{lemma}


\begin{corollary}
\( L_\text{diag}^C \notin \mathcal{L}_R \)\\
\end{corollary}

\begin{lemma}
\( L_\text{diag}^C \in \mathcal{L}_{RE} \)\\
\end{lemma}

\begin{corollary}
\( L_\text{diag}^C \in \mathcal{L}_{RE} - \mathcal{L}_R \Rightarrow \mathcal{L}_R \subsetneq \mathcal{L}_{RE} \)
\end{corollary}

\subsubsection{Rekursiv aufzählbare Sprachen, die nicht rekursiv sind}

\begin{definition}
Die \textbf{universelle Sprache} ist definiert durch \[ L_U = \{ \operatorname{Kod}(M)\#w \ |\ w \in \Sigma_\text{bool}^*, w \in L(M) \} \]
\end{definition}

\begin{satz}
Es existiert die \textbf{universelle} TM \( U \), so dass \( L(U) = L_U \). Daraus folgt, dass \( L_U \in \mathcal{L}_{RE} \).\\
\end{satz}

\begin{satz}
\( L_U \notin \mathcal{L}_R \)
\end{satz}

\begin{proof}
Es reicht zu zeigen, dass \( L_\text{diag}^C \leq_R L_U \) gilt. Dies sagt aus, dass \( L_U \) mindestens so schwer ist wie \( L_\text{diag}^C \). Wir wissen bereits, dass \( L_\text{diag}^C \notin \mathcal{L}_R \) und daraus würde automatisch folgen, dass \( L_U \notin \mathcal{L}_R \).

Sei \( A \) ein Algorithmus, der \( L_U \) entscheidet. Damit bauen wir einen Algorithmus \( B \), der \( A \) verwendet, um die Sprache \( L_\text{diag}^C \) zu entscheiden. \( B \) nimmt als Eingabe \( x \in \Sigma_\text{bool}^* \) entgegen. Diese Eingabe wird zuerst an das Unterprogramm \( C \) weitergeleitet. \( C \) berechnet an welcher Stelle \( i \) das Wort \( x = w_i \) in kanonischer Reihenfolge steht. Mittels dieser Information erzeugt es zusätzlich \( \operatorname{Kod}(M_i) \), also die \( i \)-te TM. Das Resultat von \( C \) ist somit \(\operatorname{Kod}(M_i)\#w_i\). Diese Ausgabe wird als Eingabe an \( A \) geleitet. Wir wissen, dass \( A \) entscheidet, ob \( M_i \) auf \( w_i \) hält. Nach der Vorgabe hält \( A \) auf der gegebenen Eingabe. Daraus folgt, dass auch \( B \) hält, da sowohl \( C \) als auch \( A \) Algorithmen sind. Weiter ist offensichtlich, dass \( L(B) = L_\text{diag}^C \). Somit ist \( L_U \) mindestens so schwer wie \( L_\text{diag}^C \). Daraus folgt, dass \( L_U \notin \mathcal{L}_R \).
\end{proof}

\begin{proof}[Beweis mittels EE-Reduktion]
Es gilt nach unserem bisherigen Wissensstand: \( L_\text{diag}^C \leq_{EE} L_U \Rightarrow L_\text{diag}^C \leq_R L_U \). Somit zeigen wir nun den Beweis mittels EE-Reduktion. Sei \( M \) eine TM, die eine Abbildung \( f_M: \Sigma_\text{bool} \to \{0, 1, \# \}^* \) berechnet, so dass \[ x \in L_\text{diag}^C \Leftrightarrow f_M(x) \in L_U. \]
M arbeitet nun wie folgt. Für eine Eingabe \( x \) berechnet \( M \) zuerst ein \( i \), so dass \( x = w_i \) (\(x\) ist das \(i\)-te Wort in kanonischer Reihenfolge). Danach berechnet \( M \) die Kodierung \(\operatorname{Kod}(M_i)\) der \(i\)-ten TM. \( M \) hält mit dem Inhalt \(\operatorname{Kod}(M_i) \# w_i\) auf dem Band. Weil \( x = w_i \), ist es nach Definition von \( L_\text{diag}^C \) offensichtlich, dass
\begin{align*}
x = w_i \in L_\text{diag}^C &\Leftrightarrow M_i \text{ akzeptiert } w_i\\
&\Leftrightarrow w_i \in L(M_i)\\
&\Leftrightarrow \operatorname{Kod}(M_i) \# w_i \in L_U
\end{align*}

\end{proof}

\subsubsection{Halteproblem}
\begin{definition}
Das \textbf{Halteproblem} ist das Entscheidungsproblem \\
\( (\{0, 1, \#\}^*, L_H) \), wobei
\[
L_H = \{ \operatorname{Kod}(M) \# x \ |\ x \in \Sigma_\text{bool}^* \land M \text{ hält auf } x \}
\]\\
\end{definition}

\begin{satz}
\( L_H \notin \mathcal{L}_R \)\\
\end{satz}

\begin{proof}[Beweis auf der Ebene von Programmen]
Wir möchten zeigen, dass \( L_U \leq_R L_H \) gilt. Wir zeigen also, dass \( L_H \) mindestens so schwierig ist wie \( L_U \).

Sei \( L_H \in \mathcal{L}_R \). Daher existiert ein Algorithmus \( H \), der \( L_H \) akzeptiert. Wir bauen nun damit den Algorithmus \( U \) wie folgt. Als Eingabe erhalten wir das Wort \( w \). Ein Unterprogramm \( C \) prüft, ob \(w = x \# y \) mit \( x = \operatorname{Kod}(M) \) für eine TM \( M \) ist. Falls nicht, so verwirft \( C \) die Eingabe, was zum Verwerfen der Eingabe in \( U \) folgt.

Falls die Eingabe das gewünschte Format hat, wird diese weiter an unser Unterprogramm \( H \) geleitet. \( H \) prüft nun, ob die TM \(M\) auf der Eingabe \( y \) hält. Falls nicht, so wissen wir, dass \( y \notin L(M) \Rightarrow x \# y \notin L_U \) und verwerfen die Eingabe in \( U \). Hält hingegen \( M \), so wissen wir, dass \( M \) nach \textbf{endlich} vielen Schritten \( y \) akzeptiert hat oder nicht. Somit übergeben wir die Eingabe einem weiteren Unterprogramm \( S \), welches \( M \) auf der Eingabe \( y \) simuliert. Falls \( M \) in  \( q_\text{accept} \) landet, so gilt \( x \# y \in L_U \), sonst nicht.

Es ist offensichtlich, dass \( L(U) = L_U \). Daraus folgt: \( L_U \leq_R L_H \land L(U) \notin \mathcal{L}_R \Rightarrow L_H \notin \mathcal{L}_R \)

\end{proof}

\subsubsection{Leere Sprache}
\begin{definition}
\[
L_\text{empty} = \{ \operatorname{Kod}(M) \ |\ L(M) = \emptyset \}
\]

Enthält die Kodierung aller TM, die die leere Menge akzeptieren.\\
\end{definition}

\begin{corollary}
\[
L_\text{empty}^C = \{ x \in \Sigma_\text{bool}^* \ |\ (x \notin \operatorname{Kod}(M') \text{ für alle TM } M') \lor (x = \operatorname{Kod}(M) \land L(M) \notin \emptyset) \}
\]
\end{corollary}

\begin{lemma}
\( L_\text{empty}^C \in \mathcal{L}_{RE} \)
\end{lemma}

\todo[inline]{Seite 182...}
\chapter{Grammatiken und Chomsky-Hierarchie}
Buchkapitel 10.1 bis und mit 10.3 sind Teil des ersten Selbststudiums

\section{Einleitung}
Endliche Automaten (EA) und Turingmaschinen (TM) erlauben es unendliche Objekte wie Sprachen und Mengen in endlicher Form zu beschreiben. Grammatiken sind eine weitere Möglichkeit Sprachen in endlicher und eindeutiger Weise zu beschreiben.

\section{Grammatiken}

\todo[inline]{Beispiel 10.1?}

\begin{definition}
Eine Grammatik $G$ ist ein 4-Tupel $G = (\Sigma_N, \Sigma_T, P, S)$ wobei die Bedeutung folgende ist:
\begin{itemize}
  \item Das Alphabet $\Sigma_N$ ist das \textbf{Nichtterminalalphabet}. Die Symbole aus $\Sigma_N$ nennt man \textbf{Nichtterminale}.
  \item Das Alphabet $\Sigma_T$ ist das \textbf{Terminalalphabet}. Die Symbole aus $\Sigma_T$ nennt man \textbf{Terminalsymbole}.
  \item $S \in \Sigma_N$ und ist das \textbf{Startsymbol}. Somit muss die Generierung eines Wortes jeweils mit dem Wort $w = S$ beginnen.
  \item $P$ ist eine endliche Teilmenge von $\Sigma^* \Sigma_N \Sigma^* \times \Sigma^*$ wobei $\Sigma = \Sigma_N \cup \Sigma_T$ ist. Die Elemente von $P$ heissen \textbf{Regeln}. Statt $(\alpha, \beta) \in P$ zu schreiben, wird meist $\alpha \to_G \beta$ geschrieben. Wobei die Bedeutung wie folgt ist: $\alpha$ kann in $G$ durch $\beta$ ersetzt werden.\\
\end{itemize}
\end{definition}

\begin{remark}
In dieser Vorlesung gilt:
\begin{itemize}
  \item Kleinbuchstaben $a, b, c, d, e$ und Ziffern werden für Terminalsymbole verwendet.
  \item Grossbuchstaben $A, B, C, D, X, Y, Z$ werden für Nichtterminale verwendet.
  \item Mit Kleinbuchstaben $u, v, w, x, y, z$ werden Wörter über $\Sigma_T$ bezeichnet.
  \item Mit griechischen Kleinbuchstaben ($\alpha, \beta, \gamma, \ldots$) werden beliebige Wörter über $\Sigma = \Sigma_T \cup \Sigma_N$ bezeichnet.\\
\end{itemize}
\end{remark}

\begin{remark}
Folgende Dinge sind zu beachten:
\begin{itemize}
  \item Es hat jeweils $\Sigma_N \cap \Sigma_T = \emptyset$ zu gelten.
  \item Durch die Anforderung $\alpha \in \Sigma^* \Sigma_N \Sigma^*$ muss $\alpha$ mindestens ein Nichtterminal enthalten.\\
\end{itemize}
\end{remark}

\begin{definition}
Sei $\gamma, \delta \in (\Sigma_N \cup \Sigma_T)^*$. $\delta$ ist aus $\gamma$ in einem Ableitungsschritt in $G$ ableitbar, $\gamma \Rightarrow_G \delta$, genau dann, wenn $\omega_1, \omega_2 \in (\Sigma_N \cup \Sigma_T)^*$ und eine Regel $(\alpha, \beta) \in P$ existieren, so dass gilt: $\gamma = \omega_1 \alpha \omega_2$ und $\delta = \omega_1 \beta \omega_2$.\\

$\delta$ ist aus $\gamma$ ableitbar in $G$, $\gamma \Rightarrow_G^* \delta$, genau dann wenn
\begin{itemize}
  \item entweder $\gamma = \delta$,
  \item oder für ein $n \in \N - \{0\}$ und $n+1$ Wörter $\omega_1, \omega_2, \ldots, \omega_n \in (\Sigma_N \cup \Sigma_T)^*$ existieren, so dass $\gamma = \omega_0, \delta = \omega_n$ und $\omega_i \Rightarrow_G \omega_{i+1}$ für $i = 0, 1, 2, \ldots, n - 1$.
\end{itemize}
In anderen Worten: Falls $\gamma \Rightarrow_G^* \delta$ gilt, so gibt es eine Folge von Ableitungsschritten, die bei $\gamma = \omega_1$ beginnt und bei $\delta = \omega_n$ endet.

Somit ist $\Rightarrow_G^*$ die reflexive und transitive Hülle von $\Rightarrow_G$.\\
\end{definition}

\begin{definition}
Falls $\omega \in \Sigma_T^*$ und $S \Rightarrow_G^* \omega$ gilt, dann sagt man, dass $\omega$ von $G$ erzeugt wird. Die von $G$ erzeugte Sprache ist somit $L(G) = \{\omega \in \Sigma_T^* | S \Rightarrow_G^* \omega\}$.\\
\end{definition}

\begin{remark}
Sei $\alpha \Rightarrow_G^i \beta$ eine Ableitung im Sinne von $\alpha \Rightarrow_G^* \beta$, die aus genau $i$ Schritten besteht.
\end{remark}

Grammatiken sind nichtdeterministische Erzeugungsmechanismen für Sprachen. Dies kommt daher, dass es mehrere gleiche linke Seiten geben darf und die Wahl der Anwendung einer dieser nicht festgelegt ist.

\todo[inline]{Beweistypen einbauen}

\section{Chomsky-Hierarchie}
\begin{definition}
Sei $G = (\Sigma_N, \Sigma_T, P, S)$ eine Grammatik
\begin{itemize}
  \item $G$ ist eine \textbf{Typ-0-Grammatik}. Die Typ-0-Grammatik ist die Klasse aller uneingeschränkten Grammatiken.
  \item $G$ ist \textbf{kontextsensitiv} oder \textbf{Typ-1-Grammatik}, falls $\forall (\alpha, \beta) \in P: \, |\alpha| \leq |\beta|$ gilt. Es gibt somit nicht die Möglichkeit ein Teilwort $\alpha$ durch ein kürzeres Teilwort $\beta$ zu ersetzen.
  \item $G$ ist \textbf{kontextfrei} oder \textbf{Typ-2-Grammatik}, falls $\forall (\alpha, \beta) \in P: \, \alpha \in \Sigma_N \land \beta \in (\Sigma_N \cup \Sigma_T)^*$ gilt. Alle Regeln haben also die Grundform $X \to \beta$ für ein Nichtterminal $X \in \Sigma_N$.
  \item $G$ ist \textbf{regulär} oder \textbf{Typ-3-Grammatik}, falls $\forall (\alpha, \beta) \in P: \, \alpha \in \Sigma_N \land \beta \in (\Sigma_T^* \cdot \Sigma_N \cup \Sigma_T^*)$. Somit haben alle Regeln einer regulären Grammatik entweder die Form $X \to u$ oder $X \to uY$ für $u \in \Sigma_T^*$ und $X, Y \in \Sigma_N$. Auch ist anzumerken, dass das Nichtterminal immer ganz rechts auf der rechten Seite zu stehen hat.\\
\end{itemize}
\end{definition}

\begin{remark}
Eine Sprache ist vom Typ $i$ ($i = 0, 1, 2, 3$), falls sie durch eine Typ-$i$-Grammatik erzeugt werden kann.\\
\end{remark}

Kontextfreie Sprachen haben die Eigenschaft, dass Nichtterminale unabhängig von den benachbarten Symbolen ersetzt werden können.

\section{Reguläre Grammatiken und endliche Automaten}

\begin{lemma}
$\mathcal{L}_3$ enthält alle endlichen Sprachen.\\
\end{lemma}

\begin{lemma}
$\mathcal{L}_3$ ist abgeschlossen bezüglich der Vereinigung. Somit gilt für alle Sprachen $L_1, L_2 \in \mathcal{L}_3: \, L_1 \cup L_2 \in \mathcal{L}_3$.\\
\end{lemma}

\begin{lemma}
$\mathcal{L}_3$ ist abgeschlossen bezüglich der Konkatenation. Somit gilt für alle Sprachen $L_1, L_2 \in \mathcal{L}_3: \, L_1 \cdot L_2 \in \mathcal{L}_3$.\\
\end{lemma}

\begin{satz}
Zu jedem endlichen Automaten (EA) $A$ existiert eine reguläre Grammatik $G$ mit $L(A) = L(G)$.\\
\end{satz}

\begin{definition}
Eine reguläre Grammatik (Typ-3-Grammatik) $G = (\Sigma_N, \Sigma_T, P, S)$ heisst \textbf{normiert}, wenn alle Regeln der Grammatik nur eine der folgenden drei Formen haben:
\begin{itemize}
  \item $S \to \lambda$, wobei $S$ das Startsymbol ist
  \item $A \to a$, wobei $A \in \Sigma_N$ und $a \in \Sigma_T$
  \item $B \to bC$, wobei $B, C \in \Sigma_N$ und $b \in \Sigma_T$\\
\end{itemize}
\end{definition}

\begin{lemma}
Für jede reguläre Grammatik $G$ existiert eine äquivalente \textbf{normierte} reguläre Grammatik $G'$.\\
\end{lemma}

Eine nicht normierte reguläre Grammatik $G$ kann dabei wie folgt in eine äquivalente und normierte reguläre Grammatik $G'$ überführt werden:
\begin{itemize}
  \item \textbf{Kettenregeln:} Regeln der Form $X \to Y, \, X, Y \in \Sigma_N$ enden nach endlich vielen Ableitungsschritten in $\alpha \in (\Sigma_T^* \cup \Sigma_T^+ \cdot \Sigma_N)$. Somit ersetzen wir $X \to Y$ durch $X \to \alpha$.
  \item Alle Regeln der Form $A \to \lambda$ für $A \in \Sigma_N - \{S\}$. In einer normierten regulären Grammatik darf nur aus dem Startsymbol das leere Wort abgeleitet werden. Dazu betrachten wir die Regeln $B \to \omega A, \, A \to \lambda$ womit durch das hinzufügen der Regel $B \to \omega$ und entfernen der Regel $A \to \lambda$ das Problem gelöst wird.\\
\end{itemize}

\begin{satz}
$\mathcal{L}_3 = \mathcal{L}(EA)$
\end{satz}



\chapter{Hilfreiches und Verschiedenes}
\section{Hamiltonischer Kreis}
Ein \textbf{Hamiltonischer Kreis} eines Graphen $G$ ist ein geschlossener Weg, der jeden Knoten von $G$ genau einmal enthält.

\section{Traveling Salesman Problem (TSP)}
Fragestellung: Gegeben ist eine Menge von Städten und der Reisedistanz zwischen diesen Städten. Gesucht ist die kürzeste Route mit welcher alle Städte genau einmal besucht werden und mit der man wieder beim Anfang landet.

Es handelt sich um ein NP-vollständiges Problem.

Zur Lösung dieses Problems wird der kürzeste Hamiltonsche Kreis gesucht, der am gegebenen Startpunkt beginnt.

\section{Minimum Vertex Cover Problem (MIN-VCP)}
Gesucht wird die minimale Knotenmenge (\ref{sec:knotenueberdeckung}), die eine Knotenüberdeckung eines Graphen ist.

\section{Maximaler Schnitt (MAX-CUT)}
\todo[inline]{Seite 257}

\section{MAX-SAT}
\todo[inline]{Seite 256}

\section{Sprachklassen}
\subsection{Klassenübersicht}
\todo[inline]{Übersicht und Bedeutung}

\subsection{Sprachenzugehörigkeit}
\todo[inline]{Übersicht welche Sprachen zu welcher Klasse gehören}

\section{Arithmetische Operationen auf MTMs}
Sei hier jeweils \(M\) eine MTM. Zu dieser MTM gibt es immer eine äquivalente 1-Band-TM (die Definition von Platzkonstruierbarkeit verlangt zum Beispiel eine 1-Band-TM). Stehe auf dem ersten Arbeitsband \(0^n\) und auf dem zweiten Arbeitsband \(1^m\). Das dritte Arbeitsband soll \(0^{n\ \square\ m}\) enthalten, wobei \(\square \in \{+, -, \times, \div, \ldots\}\).

\subsection{Addition}
Auf dem dritten Arbeitsband soll \(0^{n + m}\) stehen.

Dazu liest \(M\) das erste Arbeitsband und schreibt auf das dritte eine \(0\) für jede gelesene \(0\). Hat \(M\) die letzte \(0\) gelesen, so beginnt \(M\) mit dem einlesen des zweiten Arbeitsbandes. Für jede gelesene \(1\) schreibt \(M\) eine \(0\) auf das dritte Arbeitsband. Danach geht \(M\) in \(q_\text{accept}\) über.

Auf dem dritten Arbeitsband steht nun \(0^{n + m}\).

\subsection{Subtraktion}
Auf dem dritten Arbeitsband soll \(0^{n - m}\) stehen, wobei wir hier von \(n \geq m\) ausgehen.

Dazu schreibt \(M\) eine \(0\) auf das dritte Arbeitsband für jede gelesene \(0\) auf dem ersten Arbeitsband. Danach liest \(M\) das zweite Arbeitsband und löscht von rechts nach links je eine \(0\) für jede gelesene \(1\) auf dem zweiten Arbeitsband. Danach geht \(M\) in \(q_\text{accept}\) über.

Auf dem dritten Arbeitsband steht nun \(0^{n - m}\).

Falls \(n < m\), so muss eine zusätzliche Abbruchbedingung hinzugefügt werden.

\subsection{Multiplikation}
Auf dem dritten Arbeitsband soll \(0^{n \times m}\) stehen.

\(M\) liest schrittweise die \(0\)-en auf dem ersten Arbeitsband. Für jede gelesene \(0\) liest \(M\) alle \(1\)-en vom zweiten Arbeitsband und für jede gelesene \(1\) schreibt es eine \(0\) auf das dritte Arbeitsband. Nachdem alle \(1\) gelesen sind, fährt \(M\) den Lese-/Schreibkopf an den Anfang vom zweiten Arbeitsband, bevor es das nächste Zeichen auf dem ersten Arbeitsband verarbeitet. Sind alle \(0\)-en auf dem ersten Arbeitsband gelesen, so geht \(M\) in \(q_\text{accept}\) über.

Auf dem dritten Arbeitsband steht nun \(0^{n \times m}\).

\subsection{Division}
Auf dem dritten Arbeitsband soll \(0^{\lfloor n \div m \rfloor}\) stehen.

Dazu liest \(M\) für jede \(1\) auf dem zweiten Arbeitsband eine \(0\) auf dem ersten Arbeitsband. Ist \(M\) am Ende des zweiten Arbeitsbandes, so schreibt \(M\) eine \(0\) auf das dritte Arbeitsband. \(M\) fährt den Lese-/Schreibkopf auf dem zweiten Arbeitsband zurück und wiederholt das vorgehen, bis keine \(0\) mehr gelesen werden kann auf dem ersten Arbeitsband. In diesem Fall geht \(M\) in \(q_\text{accept}\) über.

Auf dem dritten Arbeitsband steht nun \(0^{\lfloor n \div m \rfloor}\).

Falls \(0^{\lceil n \div m \rceil}\) gesucht ist, so schreibt man immer zuerst die \(0\) auf das dritte Arbeitsband und liest dann so viele \(0\)-en auf dem ersten Band, wie \(1\)-en auf dem zweiten Arbeitsband.

\subsection{Logarithmus zur Basis 2}
Auf dem dritten Arbeitsband soll \(0^{\lceil \log_2(n+1) \rceil}\) stehen.

\(M\) liest das zweite Eingabeband. Für jede gelesene \(0\) schreibt \(M\) die Position auf das dritte Arbeitsbands. Dies macht sie wie folgt: \(M\) initialisiert das dritte Arbeitsband mit einer \(0\). Für jede gelesene \(0\) vom zweiten Arbeitsband addiert \(M\) eine \(1\) zum Wert, der auf dem dritten Arbeitsband kodiert ist. Hat \(M\) alle \(0\)-en auf dem zweiten Eingabeband gelesen, so ist auf dem dritten Arbeitsband die Länge des Wortes binär kodiert. Dies entspricht gerade \(\lceil \log_2(n+1) \rceil = \operatorname{Bin}(n)\).

Somit steht auf dem dritten Arbeitsband nun \(0^{\lceil \log_2(n+1) \rceil}\).

\section{Formeln}
\subsection{Summenformel für Anzahl Programme}
\[
\sum_{i = k}^n 2^i = 2^{n+1} - 2^k
\]

\chapter{Vorlesung}
\section{Vorlesung vom 17.09.2013}
\begin{itemize}
  \item Zwei Prüfungen unter dem Semester: Mitte Semester und zweite Woche vor Semesterende. Sessionsprüfung muss man so oder so gehen. Durchschnitt der Semesterprüfungen sind die minimale Abschlussnote. Sessionsprüfung kann diese Note nur verbessern. Für Teilnahme an der Semesterprüfungen muss man den Grossteil der Serien gelöst haben
  \item Einführung in die Geschichte der Mathematik und die Entstehung der Informatik. Was unterscheidet die Informatik von anderen Wissenschaften?
  \item Alphabet, Wort und Wortlänge eingeführt
  \item Buchthemen übersprungen: Kodierung von Zahlen, Graphen und anderen Dingen als Wort
  \item Konkatenation eingeführt
\end{itemize}

\section{Vorlesung vom 20.09.2013}
\begin{itemize}
  \item $(\Sigma^*, \cdot)$ ist ein Monoid
  \item Teilwörter eingeführt
  \item Beispiel: gegeben ein Wort der Länge $k$, wie viele unterschiedliche Teilwörter gibt es?
  \item $w \in \Sigma^*, |w|_a$ eingeführt (Anzahl Vorkommen von $a$ in $w$)
  \item Kanonische Ordnung nochmals kurz repetiert
  \item Sprachen eingeführt
  \item Potenznotation bei Buchstaben eingeführt: $x^2 = xx = x \cdot x$
  \item Konkatenation von Sprachen eingeführt
  \item Potenzennotation bei Sprachen eingeführt
  \item Kleensche Stern eingeführt
  \item Mengenoperationen auf Sprachen
  \begin{itemize}
    \item Beweis für $L_1 L_2 \cup L_1 L_3 = L_1 (L_2 \cup L_3)$
    \item $L_1 L_2 \cap L_1 L_3 \not= L_1 (L_2 \cap L_3)$ erklärt
  \end{itemize}
  \item Homomorphismus eingeführt
  \item Mit Einführung in Algorithmische Probleme begonnen
\end{itemize}


\end{document}
