\chapter{Vorlesung}
\section{Vorlesung vom 17.09.2013}
\begin{itemize}
  \item Zwei Prüfungen unter dem Semester: Mitte Semester und zweite Woche vor Semesterende. Sessionsprüfung muss man so oder so gehen. Durchschnitt der Semesterprüfungen sind die minimale Abschlussnote. Sessionsprüfung kann diese Note nur verbessern. Für Teilnahme an der Semesterprüfungen muss man den Grossteil der Serien gelöst haben
  \item Einführung in die Geschichte der Mathematik und die Entstehung der Informatik. Was unterscheidet die Informatik von anderen Wissenschaften?
  \item Alphabet, Wort und Wortlänge eingeführt
  \item Buchthemen übersprungen: Kodierung von Zahlen, Graphen und anderen Dingen als Wort
  \item Konkatenation eingeführt
\end{itemize}

\section{Vorlesung vom 20.09.2013}
\begin{itemize}
  \item $(\Sigma^*, \cdot)$ ist ein Monoid
  \item Teilwörter eingeführt
  \item Beispiel: gegeben ein Wort der Länge $k$, wie viele unterschiedliche Teilwörter gibt es?
  \item $w \in \Sigma^*, |w|_a$ eingeführt (Anzahl Vorkommen von $a$ in $w$)
  \item Kanonische Ordnung nochmals kurz repetiert
  \item Sprachen eingeführt
  \item Potenznotation bei Buchstaben eingeführt: $x^2 = xx = x \cdot x$
  \item Konkatenation von Sprachen eingeführt
  \item Potenzennotation bei Sprachen eingeführt
  \item Kleensche Stern eingeführt
  \item Mengenoperationen auf Sprachen
  \begin{itemize}
    \item Beweis für $L_1 L_2 \cup L_1 L_3 = L_1 (L_2 \cup L_3)$
    \item $L_1 L_2 \cap L_1 L_3 \not= L_1 (L_2 \cap L_3)$ erklärt
  \end{itemize}
  \item Homomorphismus eingeführt
  \item Mit Einführung in Algorithmische Probleme begonnen
\end{itemize}

\section{Vorlesung vom 24.09.2013}
\begin{itemize}
  \item Entscheidungsproblem
  \item Algorithmus um ein Wort $w \in \Sigma^*$ auszugeben ohne Eingabe (Aufzählungsalgorithmus)
  \item offene Frage: Wie misst man den Informationsgehalt in Texten/Wörtern
  \item Shannon Entropie angesprochen (wahrscheinlich als Exkurs)
  \item Komprimierung/Kodierung zur Messung des Informationsgehalts angesprochen, gezeigt wieso es sich nicht wirklich anbietet
  \item Kolmogorov-Komplexität eingeführt
\end{itemize}

\section{Vorlesung vom 27.09.2013}
\begin{itemize}
  \item Was passiert mit der Kolmogorov-Komplexität, wenn eine andere Programmiersprache verwendet wird?
  \item Definition des ``Zufalls" durch Kolmogorov-Komplexität. Es geht dabei nicht darum, wie wahrscheinlich/zufällig ein Ereignis ist (Wahrscheinlichkeit), sondern wie zufällig ein Objekt ist, dass wir vor uns haben. Ein Objekt ist dabei zufällig, wenn man es nur durch sich selbst beschrieben werden kann und nicht durch eine kürzere Form.
  \item Primzahlensatz mit Kolmogorov-Komplexität
\end{itemize}
