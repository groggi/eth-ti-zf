\chapter{Vorlesung}
\section{Vorlesung vom 17.09.2013}
\begin{itemize}
  \item Zwei Prüfungen unter dem Semester: Mitte Semester und zweite Woche vor Semesterende. Sessionsprüfung muss man so oder so gehen. Durchschnitt der Semesterprüfungen sind die minimale Abschlussnote. Sessionsprüfung kann diese Note nur verbessern. Für Teilnahme an der Semesterprüfungen muss man den Grossteil der Serien gelöst haben
  \item Einführung in die Geschichte der Mathematik und die Entstehung der Informatik. Was unterscheidet die Informatik von anderen Wissenschaften?
  \item Alphabet, Wort und Wortlänge eingeführt
  \item Buchthemen übersprungen: Kodierung von Zahlen, Graphen und anderen Dingen als Wort
  \item Konkatenation eingeführt
\end{itemize}
